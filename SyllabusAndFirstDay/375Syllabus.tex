%http://physics.weber.edu/schroeder/whoandwhy.html

% good blog for python MC http://blog.smellthedata.com/

% http://www.physics.rutgers.edu/ugrad/351/ is a nice syllabus for a
% 2day/wk class.

% and another
% http://www.physics.umd.edu/studinfo/courses/Phys404/einstein/fall12/syllabus.html

% site with exams,
% etc. http://www.colorado.edu/physics/phys4230/phys4230_sp03/index.html 

% Recommended problems http://physics.weber.edu/thermal/recprobs.html

% Course plans http://physics.weber.edu/thermal/courseplans.html

% http://pages.physics.cornell.edu/~sethna/StatMech/

% http://www.physics.utah.edu/~rogachev/7310/7310.htm

% http://webcache.googleusercontent.com/search?q=cache:http://nile.physics.ncsu.edu/py722/f09/

% http://pubs.acs.org/doi/abs/10.1021/ci300470t

% http://jcp.aip.org/resource/1/jcpsa6/v137/i23/p234104_s1?isAuthorized=no

% http://www.technologyreview.com/view/429561/the-measurement-that-would-reveal-the-universe-as-a-computer-simulation/

% http://ipparco.roma1.infn.it/~giovanni/gpub.html#E

% iPhone MD http://mason.gmu.edu/~bkim14/i2dmd.htm

% Negative Temperature Baez
% https://plus.google.com/117663015413546257905/posts/KTeKM6SffYT
% and http://math.ucr.edu/home/baez/physics/ParticleAndNuclear/neg_temperature.html

% From Baez:
% I've never taught thermodynamics and stat mech, but if I did I'd want
% to take a modern approach starting with a little probability theory,
% then going to statistical mechanics via the maximum entropy principle,
% and then to thermodynamics.   
% 
% I think most students hate the technicalities that dominate
% thermodynamics before one explains the subject using stat mech: the
% pile of concepts like entropy, Gibbs free energy and enthalpy, and the
% Maxwell relations tying them all together, seem unintuitive and hard
% to remember.   It would be much nicer to start with the entropy of a
% probability distribution, argue that it's a measure of 'information we
% don't know', and say that the basic principle in any situation is to
% choose the maximum entropy probability distribution consistent with
% our knowledge.  Getting from here to the Maxwell distribution and then
% the ideal gas laws could be quite inspiring! 
% 
% But, having never tried this with students, much less undergraduates,
% I can't say how to make it work.  Putting in too much math or too much
% philosophy might backfire. 
% 
% And since I've never written about fluctuation-dissipation theorems, I
% can never remember how they fit together: the reason I write so much
% is that I can only work out a clear picture of a subject by trying to
% explain it.  That's when I notice all the gaps in my understanding. 
% 
% So, all I can say is: please make your course notes available to the world!


\documentclass[12pt]{article}
%%%%%%%%%%%%%%%%%%%%%%%%%%%%%%%%%%%%%%%%%%%%%%%%%%%%%%%%%%%%%
\input{macros}
%%%%%%%%%%%%%%%%%%%%%%%%%%%%%%%%%%%%%%%%%%%%%%%%%%%%%%%%%%%%%
\usepackage[table]{xcolor}
\usepackage{multirow}

\usepackage{hyperref}
\hypersetup{
    colorlinks=true,
    linkcolor=blue,
    filecolor=magenta,      
    urlcolor=blue,
  }

\usepackage{float}
\restylefloat{table}

\pagestyle{empty}

\renewcommand{\thefootnote}{\fnsymbol{footnote}}
\begin{document}

\begin{center}
{\bf Physics 375: Thermal \& Statistical Physics; T$\Theta$
  12:40PM-4:00PM; Zoom
}
\end{center}

\setlength{\unitlength}{1in}

\hrule

\vskip.15in 
\noindent\textbf{Instructor:} Michael Lerner,  CST 213, Phone: 727-LERNERM
\vskip.15in
\noindent\textbf{Office Hours:} MWF 9-10 and 11-12, T$\Theta$ 9:00-11:00, 14:00-15:00 and by  appointment. %I also have an open-door policy, and you're
%encouraged to stop in to ask questions whenever my door is open. That's
%most of the time.

\vskip.15in 
\noindent\textbf{When and where:} Synchronous class meetings will be MWF 10:00-11:00 and T$\Theta$ 13:00-14:00. All meetings take place on Zoom: \url{https://us02web.zoom.us/j/88190538447}

\vskip.15in 
\noindent\textbf{Credits:} This is a four-credit course. In normal times, this is fulfilled via four hours of in-person instruction per week over a 15-week semester. During a pandemic, we'll have three hours per week of synchronous textbook-based instruction, two hours per week of synchronous project time, two hours per week of project work, and several hours per week of interactions via online discussion.

\vskip.15in 
\noindent\textbf{COVID:} Campus-wide coronavirus policies and information: \url{https://earlham.edu/coronavirus}


\vskip.15in 
\begin{description}
\item[Course goals]\ \\\vspace{-.3in}
\begin{itemize}
  \item Students will understand the basics of thermal physics: how do macroscopic quantities such as temperature relate to each other?
  \item Students will understand the statistical underpinnings of thermal physics from a molecular level, including foundational topics and modern formulations of the second ``law'' of thermodynamics.
  \item Students will understand and apply the concepts of statistical and thermal physics to a topic in their area of interest.
  \item Students will explore the wide range of applications of statistical mechanics by developing a Monte Carlo model to simulate and evaluate March Madness brackets or Pokemon Go battles.
\end{itemize}
\item[Required Textbook:] Schroeder, \textbf{Thermal Physics} It's extremely readable, and has a good selection of
real-world problems.
\item[Required Software] You'll need a spreadsheet program (either
  Excel or Google Sheets) and \textbf{The Anaconda Python
    Distribution} (\url{http://continuum.io/downloads}. We'll do several computational exercises throughout
  the class. If you have a laptop, please install the Anaconda Python
  Distribution. If you do not have a laptop, please contact me ASAP.

\item[Required Online Forums:] Moodle, and \url{piazza.com/earlham/spring2021/phys375}
\item[Prerequisite] Physics 345, modern physics.
  \vskip.15in 
\item[Communication:] My preferred mode of communication is email. I will check my email multiple times per day during the work week. If you have sent me an email and have not heard back from me within 24 hours (excluding weekends), you should email me again or drop by my office to ask your question in person. I cannot guarantee that I will regularly check my email on evenings or on weekends.

  I expect that you will check your Earlham email at least once per day. That is, 24 hours after sending an email to the class, it is my expectation that all of you will have read that email. Emailed assignments, due dates, policy changes, and other course information carry the same weight as information contained in the syllabus.

  Information relevant to the course will also be posted to the Moodle page. You are responsible for checking Moodle at the same frequency and with the same conditions applied to email communication.

\item[Grading Policy:]
  Your grade in this class will be determined by a weighted combination of your scores from multiple categories. Within each category, each assignment is weighted equally. It is extremely hard to catch up on late work in normal times (usually things are late because students are busy, and people don't tend to get less busy!). This is especially true in a pandemic. So, rather than accept late work, I will drop a roughly a week worth of your lowest grades from most categories. If you have to turn things in late, my strong suggestion is that you prioritize getting your schedule back under control. Feel free to simply \textit{not do} things that will not be graded. Especially if it helps you keep the rest of your life under control.

%%\begin{itemize}
%%  \item Attendance/preparation/participation (including Piazza): 15\%
%%  \item Daily homework, 20\%
%%  \item Weekly homework, 20\%
%% \item Group work, 15\%
%%  \item Independent project, 20\%
%%%  \item One midterm, 10\%
%%%  \item Three in-class labs, each 4\%
%%%  \item Two midterms, each 9\%, for a total of 18\%
%%  \item Final exam, 10\%
%%  \end{itemize}\vspace{-.2in}

\setlength{\arrayrulewidth}{1mm}
\setlength{\tabcolsep}{8pt}
%\renewcommand{\arraystretch}{2.5}
{\rowcolors{2}{green!80!yellow!50}{green!70!yellow!40}
\begin{table}[H]
%\caption{Grading Breakdown}
\label{tab:my-table}
\begin{tabular}{llll}
\hline
\textbf{Assignment}                                              & \textbf{Number} & \textbf{Drop} & \textbf{Weight} \\ \hline
\textbf{Attendance/preparation/participation (including Piazza)} & 21     & 3    & 15     \\
\textbf{Daily homework }                                         & 20     & 2    & 20     \\
\textbf{Weekly homework }                                        & 6      & 1    & 20     \\
\textbf{Group work }                                             & 13     & 2    & 15     \\
\textbf{Independent project }                                    & 1      & 0    & 20     \\
\textbf{Final exam }                                              & 1      & 0    & 10     \\ \hline
\end{tabular}
\end{table}
}

\item[Academic Accommodations:] {\small \url{http://www.earlham.edu/curriculum-guide/academic-integrity/}}
 Students with a documented disability (e.g., physical, learning, psychiatric, visual, hearing, etc.) who need to arrange reasonable classroom accommodations must request accommodation memos from the Academic Enrichment Center(main floor of Lilly Library) and contact their instructors each semester. For greater success, students are strongly encouraged to visit the Academic Enrichment Center within the first two weeks of each semester to begin the process.
  
\item[Extra Credit:]
  Throughout the course, several homework problems are explicitly marked as extra credit. The other opportunities are listed here. Extra credit will not be included in grade updates, but will be included when I calculate your final grade in the course.
  \begin{itemize}
  \item 0.5\% extra credit for visiting me in my office hours in the first two weeks of class.
  \item 0.5\% extra credit to the whole class if the entire class completes their course evaluations.
  \item 0.1\% extra credit for the first person to point out any significant errors on tests or any assignment, up to a maximum of 1.0\%.\ (Spelling errors and similar mistakes do not count as significant errors unless they significantly impede your ability to do the assignment) 
    \end{itemize}

  
\item[Attendance/preparation/participation (including Piazza):] Especially in a pandemic, attendance is (1) critical (2) sometimes hard to sustain. Attendance involves attending the synchronous Zoom lectures and watching the pre-recorded videos. During class, we will work on problems together and via groups in breakout rooms. This problem solving is a critical component of the class, and it is the main reason attendance is required. We all know that circumstances sometimes make it hard or impossible to attend, so I will drop your three lowest attendance and participation grades.

  It is very important that you read the appropriate text assignments before coming to class. I will repeat and re-emphasize material from the book, but you'll find that you understand and internalize the material much better if the class discussion is your second exposure to the material. In this class in particular, there are a huge variety of applied problems that we'd love to discuss. The more you've read before class, the more we can delve into the new material in class. So, \textbf{You are required to post to the Piazza site by 9 PM the day before class on Mondays, Wednesday, and Fridays.}. Your post should include what you thought was the most interesting and important from the reading. You should mention any parts of teh reading that were particularly hard to understand. You can also comment to ask a question or answer someone else's question in lieu of mentioning the most interesting/important things. This counts for 5\% of your overall grade.

\item[Daily homework:] I will assign a small number of easy and medium-difficulty problems after each MWF lecture. You are responsible for completing these problems and submitting them on Moodle as a PDF by the start of the next lecture. I will check these problems only for effort and completion, and the expectation is that each assignment will take approximately one hour on average. The lowest 2 daily homework assignments will be dropped.

  If it is apparent you attempted all of the problems and your work is clear, you will receive full credit, even if you got the wrong answer. If you are unable to complete a problem, you are expected to provide reflection on why you were unable to complete the problem (e.g., you know that you need to use a thermodynamic identity, but were unable to find a solution using these methods). Providing these reflective comments on unfinished problems will help you improve your problem-solving process, and will save you points.

\item[Weekly Homework Assignments:] Every week on Monday, you will be assigned a homework consisting of medium and high-difficulty problems that will take more time to work through. You are responsible for submitting these problems on Moodle as a PDF by the start of lecture on the following Monday. These assignments will be graded for correctness and are the closest we will get to quizzes and tests this term, other than the Final Exam. There are a few connected groups of problems that stretch across multiple chapters, so you will want to keep your solutions from previous weeks in case they help you in future weeks. The lowest 1 weekly homework assignment will drop from your grade. Each problem will be graded on a 5-point scale whose details are shown below.

  \begin{description}
  \item[5] Correct or close to correct, with maybe a small arithmetic error
  \item[4] Correct reasoning for the solution, with maybe an algebraic error
  \item[3] Small mistake in the reasoning for the solution, but with correct follow-through or analysis
  \item[2] Large mistake (or many small ones) in the reasoning, or significantly incomplete problems
  \item[1] Some non-trivial solution was attempted
    \item[0] No attempt was made. \textbf{You get 20\% for simply writing down SOMETHING.}
    \end{description}

  \item[Group Work:] You will be broken into groups of 2 or 3 that will be chosen by the instructor. Your group is expected to find two times to meet each week, for a total of around one hour each time. In your meetings, you will work on the assigned group work problems that are posted to Moodle. These assignments are intended to give you practice with the material in an environment where you can help each other learn.

    These assignments will be graded for effort and completion. If you show up to the meeting with your group, give your best effort, and are able to complete the problems, then you should receive full credit.

    Each week, one group work assignment will be posted on Monday and will be due by the start of class on Friday, and the second group work assignment will be posted on Wednesday and will be due by the start of class the following Monday. Though you are expected to work together on the problems, each of you is responsible for submitting your own solution as a PDF on Moodle. Some level of similarity between submissions is allowed because these are group assignments, but you are not allowed to submit the same file for all group members. In short, work together to determine the correct approach, then you each write your own solution

    \item[Independent project:]The applications of modern
statistical mechanics are so broad that we cannot hope to cover even a
reasonable sampling of them in a single course. However, I don't want
you to miss out on the parts that happen to be most interesting to
you. Therefore, you'll each pick either an interesting problem to
model or an interesting technique to learn. You'll write a short paper
and present the results to the class. You'll be expected to start on
this halfway through the semester, and we'll discuss it in more detail
at that point. Sethna's book provides a wealth of such problems, and
I'm certainly available to provide background material that you may be
missing if needed.

The times of 13:00-14:00 on Tuesdays and Thursdays are reserved for project time. I will introduce the project during this time slot on Thursday of Week 1, and we'll discuss the assignment in more detail at that point. For weeks 2-6, each of you will have a 30-minute appointment with me once a week during this time slot to discuss your progress and any questions you have in a one-on-one environment. During Week 7, each of you will make a presentation about your project to the rest of the class, and attending the presentations of other students is part of your grade on the project.

Two rough drafts of your project will be due. Given the timing of the course, the first rough draft is allowed to be quite rough, but serves to ensure that you're making good progress, have some things written down, and know what you need to do to succeed.

Topics are largely at your discretion, but may include
\vspace{-0.05in}
{\small
\begin{itemize}
\item Information Theory and Statistical Mechanics
  \item Biophysical applications (protein dynamics, diffusion, etc.)
  \item Foundations of the Zeroth and Second ``Laws'' of thermodynamics
  \item Coarsening
  \item Time correlation functions
  \item Annealing
  \item Statistical Mechanics/Thermodynamics of small systems
  \item Further work with Monte Carlo simulations
  \item Thermal ratchets, theory and practice
  \item Order parameters and critical exponents (often an in-class topic!, see Sethna)
  \item Anything in Schroeder that we do not cover
  \end{itemize}
}
    \item[Final Exam:] The weekly assignments serve effectively as small-stakes quizzes/exams, and the final exam will be our only pure test in the class. We will have a clear discussion and review of topics before the exam.

%\item[Attendance Policy:] Students are expected to attend classes regularly. A student who incurs an excessive
%number of absences may have some or all of the class preparation/participation grade (10\%) deducted at the discretion of the instructor.
%\item[Piazza:]  piazza.com/earlham/spring2021/phys375 
    \item[Academic Integrity:] {\small http://www.earlham.edu/curriculum-guide/academic-integrity/} As with all Earlham courses, this course is covered by the academic integrity policy. Much of the work in this class is group-based, but not all of it. Specific expectations are discussed in class and in this syllabus, but: daily and weekly problems should be attempted individually first. You may then ask for guidance in class and in the forums. Your final solution should be written individually. Group collaboration is expected from the beginning with the group problems, and more similarity is expected between submissions.

      \item[Workload Expectations:] Under the condensed schedule Earlham is currently using, workload expectations for each 4-credit class are at least 20 hours per week per class, including time spent in lecture. Each week, you will spend about 6 hours in lecture (synchronous and asynchronous), 2 hours on group work assignments, 3 hours on daily homework assignments, 3 hours on weekly homework assignments, 3 hours on your independent project, and 3 hours reading the textbok.

Please note that ``at least 20 hours per week'' is an average across students and across weeks. Some weeks, you'll spend more. Some weeks, you'll spend less. My aim is to keep the average workload around 20 hours per week. If, however, you find yourself needing to spend significantly more than 20 hours in a week, please let me know ASAP so that I can quickly correct things.
\end{description}


%%%\noindent\textbf{Important Dates}:
%%%\begin{center} \begin{minipage}{5in}
%%%\begin{flushleft}
%%%Drop Deadline \dotfill 3/31/2017\\
%%%Project Topics Due \dotfill 4/4/2017\\
%%%First exam \dotfill 2/14/2017\\ 
%%%Second exam \dotfill 4/11/2017\\ 
%%%Final Exam, Comprehensive emphasis on later topics \dotfill 5/4/2017\\
%%%\end{flushleft}
%%%\end{minipage}
%%%\end{center}


%\noindent\textbf{Class time} There will be two class meetings
%each week. The class meetings will generally consist of lectures,
%discussion of assignments, and student presentation of assigned
%problems. The longer meetings will allow us to work through longer
%computational exercises in class.


%%%\vskip.05in\noindent\textbf{Homework} As you surely know by now, your understanding will be
%%%significantly greater if you work problems throughout the week rather
%%%than saving them all for Sunday night. So, homework will be assigned
%%%each class period. You are actively encouraged to work cooperatively on
%%%them. \textbf{Understanding someone else's solution to a problem is not
%%%  nearly as useful as being able to solve the problem yourself.}
%%%Therefore, I ask that you attempt each problem on your own before
%%%discussing it with your peers. You may then compare answers and
%%%discuss strategies, but the solution you turn in should be written
%%%entirely by you. If you are using online solutions as part of your
%%%study, you should inform me so that we can work out guidelines for
%%%such usage.
%%%
%%%Several of the problems assigned in this class are
%%%quite challenging. Others are rote computation. For the more
%%%challenging problems, my goal is to have you make the strongest
%%%possible effort towards \textbf{understanding} the solution. Thus, if
%%%you cannot fully solve the problem, say whatever you can about the way
%%%in which a solution would proceed from where you stop; say what
%%%you can about the qualitative behavior of a solution; say what you
%%%can about the physical meaning of the solution. 
%%%
%%%\textbf{Due dates} Homework assigned on Thursday is due at the start
%%%of class the following Tuesday. Homework assigned on Tuesday is due at
%%%5:00 PM (in the box outside my office) on Friday.
%%%
%%%\textbf{Resubmission} If you get a homework problem wrong, you may
%%%redo it for half of the missed credit. If you choose this option, you
%%%must have a friend grade it (using the solution set in the library)
%%%and then submit the re-graded work to me. The goal here is to
%%%encourage you to keep thinking about these problems until you
%%%understand them while still giving credit to work done on
%%%time. Resubmissions can be turned in any time before the final.
%%%
%%%\textbf{Late work} In an ideal world, all homework would be done on
%%%time.  We seem not to inhabit that world, so how will I deal with late
%%%homework?
%%%
%%%1) On the due date, submit something – whatever you have done, even if
%%%it’s only a few problems.  You may then submit additional work late.
%%%However, unless you submit something on the due day, your late work
%%%won’t be accepted at all. Solutions will be available in the library
%%%as soon as the homework is due. If you consult these while completing
%%%late work (A) you must mention that fact on your assignment (B) you
%%%must use ``real'' late days, no the ten ``free'' late days described
%%%below.
%%%
%%%2) The part of the homework that is submitted late will penalized; later work $\Rightarrow$ larger penalty. 
%%%
%%%3) During the semester your have 10 free ``late days'' for homework.
%%%Don't use them early in the semester for frivolous reasons; you may
%%%need them toward the end of the semester.   
%%%
%%%4) Each student has a maximum of 20 late days.  Once you’ve used those
%%%up, your homework will not be accepted unless it’s on time. Please put
%%%a date on any work you submit.  Late penalties are usually 10\% per
%%%day late, with Saturday/Sunday counting as one ``day.''   Homework
%%%that is more than 1 week late generally will not be accepted for
%%%grading.
%%%
%%%
%%%
%%%\newpage
%%%Homework problems will be graded on roughly the same scale as used in
%%%Physics 125 and 235:
%%%
%%%\begin{description}
%%%  \item[5] Solution is complete and well-written
%%%  \item[4] Solution is missing minor parts or some important explanations
%%%  \item[3] Solution is missing major parts and/or has few if any explanations
%%%  \item[2] At least one major portion of the problem correct
%%%  \item[1] Very little coherent initial effort was expended
%%%  \item[0] No initial solution was submitted
%%%\end{description}
%%%
\vskip.25in\noindent\textbf{Books and Resources}
\begin{description}
\item[Additional Textbooks]
 \hfill \\
\textbf{Gould and Tobochnik, Statistical and Thermal Physics} It was a
close decision between this and Schroeder. I went with Schroeder for
several reasons, but this book is free and good.
http://www.compadre.org/stp/
\\
\textbf{An Introduction to Statistical Mechanics and Thermodynamics} An 
extremely well-written modern introduction. Paced a bit faster than we'll go,
this would be a good intro grad text, or 475-level text.
\\
\textbf{Sethna, Entropy, Order Parameters, and Complexity}
This is a freely-available, modern, advanced, applied statistical
mechanics textbook. Its main strengths include the broad range of
problems (statistical mechanics, in its modern form, is an extremely
broad, applied subject) and the extremely up-to-date content. You can
download it from Sethna's website
(http://www.lassp.cornell.edu/sethna/). On the other hand, it has a
reputation as a book that's ``fantastic if you already know the
subject matter'' but difficult to learn from on your own at the
advanced undergraduate level. The plan is to supplement Schroeder with
bits and pieces of Sethna. Please start pawing through Sethna ASAP so
that you can pick out interesting topics that we may use. \hfill \\
\item[Recommended Textbooks] \hfill \\
\textbf{Kusse and Westwig, Mathematical Physics}
This text is extremely well written. It's a good reference for the
math you may have forgotten. \\
\textbf{Schey, div grad curl and all that}
This is an extremely conversational introduction to/refresher on
vector calculus. \hfill \\
\textbf{Bridging the Vector Calculus Gap} This is a fantastic resource
for vector calculus that focuses on actually using the material in
practice, rather than just learning it in a mathematical
context. http://www.math.oregonstate.edu/bridge/ 
\\
\textbf{Styer, Statistical Mechanics} I just found this recently, but
it looks like a nice attempt at integrating modern material into a
Stat Mech course. Note that it is not yet a complete
book. http://www.oberlin.edu/physics/dstyer/StatMech/book.pdf
\end{description}

\vskip.25in\noindent\textbf{Labs} Statistical Mechanics is one of the
most active areas of modern physics research. In non-pandemic times,
we supplement the class with three labs. During a pandemic, we will do
at least two of these as group projects.

\begin{description}
\item[Diffusion] Statistical mechanics is also concerned with
  predicting diffusion constants of grains of pollen floating on water
  (think Einstein's famous 1905 paper) or proteins moving about in
  cell membranes. We will team up with Adam Hoppe at South Dakota
  State University to use a web-controlled TIRF (total internal
  reflection) microscope to measure the diffusion constant of
  individual lipids (actually, we'll be looking at quantum dots
  attached to lipids) in cell membranes. This will give us a way of
  calculating Boltzmann's constant. This lab is also of interest to
  biochemistry students, so I've invited several of them to watch and
  perhaps participate.
\item[Non-equilibrium Statistical Mechanics] Think for a minute about
  moving your hand around under water. If you move your hand
  infinitely slowly (called ``quasistatically''), you would say that
  the work required to move your hand from one place (state $A$) to
  another (state $B$) is equal to the free energy difference between
  $A$ and $B$. What if you move your hand quickly? Until very recently
  (1997), the most definite general answer we could give is that the
  work required would be greater than or equal to the free energy
  difference (you'd burn up some energy in friction), but performing
  several non-equilibrium processes like this would not be able to
  tell you exactly the free energy difference between $A$ and $B$. In
  this lab, we will study one of the most shocking results of
  statistical mechanics, the Jarzynski equality, which allows us to
  average \textit{nonequilibrium} processes to determine the exact
  energy difference between two \textit{equilibrium} states. This lab
  involves computer simulations of proteins, and is normally done in two
  parts, separated by several weeks.
\item[Entropy of Unknotting] In this lab, we will model the unknotting
  of a small beaded chain via random walks, and make both quantitative
  and qualitative measurements of the entropy involved of unknotting.
\end{description}

%The entropy of unkotting is a self-directed lab. %You may begin it at
%any point after spring break.


%%%\vskip.25in\noindent\textbf{Tests}  All will be self-scheduled exams, to be picked
%%%up and turned in at the front desk of the science library. 

%%%\newpage
%%%\vskip.25in\noindent\textbf{Independent project} The applications of modern
%%%statistical mechanics are so broad that we cannot hope to cover even a
%%%reasonable sampling of them in a single course. However, I don't want
%%%you to miss out on the parts that happen to be most interesting to
%%%you. Therefore, you'll each pick either an interesting problem to
%%%model or an interesting technique to learn. You'll write a short paper
%%%and present the results to the class. You'll be expected to start on
%%%this halfway through the semester, and we'll discuss it in more detail
%%%at that point. Sethna's book provides a wealth of such problems, and
%%%I'm certainly available to provide background material that you may be
%%%missing if needed.
%%%
%%%The times of 13:00-14:00 on Tuesdays and Thursdays are reserved for project time. I will introduce the project during this time slot on Thursday of Week 1, and we'll discuss the assighment in more detail at that point. For weeks 2-6, each of you will have a 30-minute appointment with me once a week during this time slot to discuss your progress and any questions you have in a one-on-one environment. During Week 7, each of you will make a presentation about your project to the rest of the class, and attending the presentations of other students is part of your grade on the project.
%%%
%%%A rough draft of your project will be due at the end of week 4, \textbf{and the rough draft is worth 5\% of your course grade}. Given the timing of the course, the rough draft is allowed to be quite rough, but serves to ensure that you're making good progress, have some things written down, and know what you need to do to succeed.
%%%
%%%Topics are largely at your discretion, but may include
%%%
%%%\begin{itemize}
%%%  \item Information Theory and Statistical Mechanics
%%%  \item Foundations of the Zeroth and Second ``Laws'' of
%%%    thermodynamics
%%%  \item Coarsening
%%%  \item Time correlation functions
%%%   \item Annealing
%%%  \item Statistical Mechanics/Thermodynamics of small systems
%%%  \item Further work with Monte Carlo simulations
%%%  \item Thermal ratchets, theory and practice
%%%  \item Order parameters and critical exponents (often an in-class
%%%    topic!, see Sethna)
%%%    \item Anything in Schroeder that we do not cover
%%%\end{itemize}

\vskip.25in\noindent\textbf{Course Outline} 

My plan is to move fairly quickly through the first part of the book,
assuming it's mostly review. Later sections are up for discussion:
should we spend more time on heat engines, or more time on statistical
mechanics and applications?

In any case, we'll be doing several computational simulations
throughout. In order to make things standard, we'll do them all in
Python.

In addition to the focus on simulation, we'll focus more on
statistical mechanics than on thermodynamics, so we will skip straight
from Chapter 3 to Chapter 6, giving us time to set up Monte Carlo
simulations of March Madness. The syllabus is fairly flexible, but we
need to get to MD simulations \textit{before} March, and we need to
get through diffusion before the lab is scheduled. \textbf{IMPORTANT NOTE:} This year, we can replace March Madness with Pokemon Go if the class wants.

I expect this class
to be a lot of work, and an enormous amount of fun.

% Plan: Ch. 1, Ch.2, Ch. 3, Section 6.1, Section 6.2, Section 8.2
\newpage
%\begin{tabular}{p{1cm}|p{2cm}|p{2cm}|p{0.25\linewidth}|p{2cm}|p{1cm}|p{1cm}}

\setlength{\arrayrulewidth}{.4mm}
\setlength{\tabcolsep}{8pt}
{\rowcolors{2}{green!80!yellow!50}{green!70!yellow!40}
  \begin{table}[]
    \footnotesize
\caption{\footnotesize{Course Schedule (D = daily homework, W = weekly homework, G = group work}}
\label{tab:course-calendar}
%\begin{tabular}{lllllll}
\begin{tabular}{p{.03\linewidth}|p{.065\linewidth}|p{.065\linewidth}|p{0.4\linewidth}|p{.08\linewidth}|p{.085\linewidth}|p{.1\linewidth}}
\textbf{Wk} & \textbf{Date}    & \textbf{Read}        & \textbf{Topics} & \textbf{In-class probs} & \textbf{Due}    & \textbf{Project time}  \\ \hline 
1    & Feb 1  & 1.1-1.4        & What is Statistical Physics? Equilibrium; ideal gas; equipartition; heat and work  & 4, 14, 18                           & n/a          & n/a           \\
1    & Feb 3  & 1.5-1.6        & Compressive work; Heat capacities; Rates                               & 37, 45                                & D1           & n/a           \\
1    & Feb 5  & 2.1-2.3        & Two-State Systems; Einstein model of a solid; Interacting systems                   &                                           & D2, G1       & Intro \\ \hline
2    & Feb 8  & 2.1-2.3 & Two-state systems; Einstein model; interacting systems      &  16 & D3, G2, W1   & \multirow{2}{*}{Proposals}     \\
2    & Feb 10 & 2.4-2.5        & Large Systems; Ideal Gas                                                                    &                                           & D4           & Proposals     \\
2    & Feb 12 & 2.6, 3.1       & ENTROPY!; Temperature                                                     &                                           & D5, G3       & Proposals     \\ \hline
3    & Feb 15 &3.2, 3.3       & Entropy and Heat; Paramagnetism   &                                           & D6, G4, W2   & Check-in 1    \\
3    & Feb 17 &3.4, 3.5, 3.6  & Mechanical Equilibrium and Pressure; Diffusive Equilibrium and Chemical Potential        &                                           & D7           & Check-in 1    \\
3    & Feb 19 & 4.1-4.4        & Heat Engines and Refrigerators (4.1-4.2 are much more important than 4.3-4.4)     &                                 & D8, G5       & Check-in 1    \\ \hline
4    & Feb 22 & Ch. 4 &Heat engines, throttling &                                           & D9, G6, W3   & Check-in 2    \\
4    & Feb 24 &Ch. 4, 5.1 & Problem 4.12; ``derive'' F,G  &                                           & D10          & Check-in 2    \\
4    & Feb 26 & 5.2, 5.3 through p. 174 &TIs quickly, Free energy as force towards equilibrium, some phase transitions. &                                                                                     & D11, G7      & Check-in 2    \\ \hline
5    & Mar 1  & 5.3 & phase transformation of pure substances & 7                                       & D12, G8, W4  & Rough draft 1 \\
5    & Mar 3  & 6.1-6.2 &The Boltzmann Factor, Average values       &                                           & D13          & Rough draft 1 \\
5    & Mar 5  & 6.5-6.7        & Partition Functions, Free Energy and Composite Systems &                                           & D14, G9      & Rough draft 1 \\ \hline
6    & Mar 8  &  7.1-7.2        & The Gibbs Factor; Bosons and Fermions                            &                                           & D15, G10, W5 & Draft 2 \\
6    & Mar 10 &  7.3            & Degenerate Fermi Gases, Density of States                                                       &                                           & D16          & Draft 2 \\ \hline \hline
6    & Mar 12 &  7.5-7.6        & Debye Theory; Bose-Einstein Condensation                        &                                           & D17, G11     & Draft 2 \\ \hline
7    & Mar 15 & 8.2            & Ising models &                                           & D18, G12, W6 & Present \\
7    & Mar 17 & Blog &Monte Carlo March Madness &                                           & D19          & Present \\
7    & Mar 19 &5.4          &Phase transitions of mixtures &                                           & D20, G13     & Present \\ \hline
8    & Mar 22 & \multicolumn{3}{c}{FINAL EXAM, Tuesday, March 23 8:00-10:00}                                                                                                    & &n/a          
\end{tabular}
\end{table}

%% HERE IS WHAT WE LOST
%5.5, 6.3-6.4   
%8.2, Ising.pdf 
% PDFs           
%Dilute Solutions; Equipartition; Maxwell Speed Distribution                                          
%More about MC; MC Pi estimation, Monte Carlo Simulation Coding; March Madness; The new fluctuation theorems                         
%1.7  Diffusion, rates 





 %Future Michael: here is the original calendar.

%\setlength{\arrayrulewidth}{.4mm}
%\setlength{\tabcolsep}{8pt}
%{\rowcolors{2}{green!80!yellow!50}{green!70!yellow!40}
%  \begin{table}[]
%    \footnotesize
%\caption{\footnotesize{Course Schedule (D = daily homework, W = weekly homework, G = group work}}
%\label{tab:course-calendar}
%%\begin{tabular}{lllllll}
%\begin{tabular}{p{.03\linewidth}|p{.065\linewidth}|p{.065\linewidth}|p{0.4\linewidth}|p{.08\linewidth}|p{.085\linewidth}|p{.1\linewidth}}
%\textbf{Wk} & \textbf{Date}    & \textbf{Read}        & \textbf{Topics} & \textbf{In-class probs} & \textbf{Due}    & \textbf{Project time}  \\ \hline 
%1    & Feb 1  & 1.1-1.4        & What is Statistical Physics? Equilibrium; ideal gas; equipartition; heat and work  & 4, 14, 18                           & n/a          & n/a           \\
%1    & Feb 3  & 1.5-1.6        & Compressive work; Heat capacities; Rates                               & 37, 45                                & D1           & n/a           \\
%1    & Feb 5  & 2.1-2.3        & Two-State Systems; Einstein model of a solid; Interacting systems                   &                                           & D2, G1       & Intro \\ \hline
%2    & Feb 8  & 2.1-2.3 & Two-state systems; Einstein model; interacting systems      &  16 & D3, G2, W1   & \multirow{2}{*}{Proposals}     \\
%2    & Feb 10 & 2.4-2.5        & Large Systems; Ideal Gas                                                                    &                                           & D4           & Proposals     \\
%2    & Feb 12 & 2.6, 3.1       & ENTROPY!; Temperature                                                     &                                           & D5, G3       & Proposals     \\ \hline
%3    & Feb 15 &3.2, 3.3       & Entropy and Heat; Paramagnetism   &                                           & D6, G4, W2   & Check-in 1    \\
%3    & Feb 17 &3.4, 3.5, 3.6  & Mechanical Equilibrium and Pressure; Diffusive Equilibrium and Chemical Potential        &                                           & D7           & Check-in 1    \\
%3    & Feb 19 & 4.1-4.4        & Heat Engines and Refrigerators (4.1-4.2 are much more important than 4.3-4.4)     &                                 & D8, G5       & Check-in 1    \\ \hline
%4    & Feb 22 & 5.1-5.2        & Free energy available as work; Free Energy as a force towards equilibrium               &                                           & D9, G6, W3   & Check-in 2    \\
%4    & Feb 24 & 5.3            & Phase Transformations of Pure Substances             &                                           & D10          & Check-in 2    \\
%4    & Feb 26 & 6.1-6.2        & The Boltzmann Factor, Average values      &                                                                                     & D11, G7      & Check-in 2    \\ \hline
%5    & Mar 1  &6.5-6.7        & Partition Functions, Free Energy and Composite Systems Also catch up       & 7                                       & D12, G8, W4  & Rough draft 1 \\
%5    & Mar 3  & 7.1-7.2        & The Gibbs Factor; Bosons and Fermions                           &                                           & D13          & Rough draft 1 \\
%5    & Mar 5  & 7.3            & Degenerate Fermi Gases, Density of States                                                      &                                           & D14, G9      & Rough draft 1 \\ \hline
%6    & Mar 8  & 7.5-7.6        & Debye Theory; Bose-Einstein Condensation                       &                                           & D15, G10, W5 & Draft 2 \\
%6    & Mar 10 & 8.2            & Ising models               &                                           & D16          & Draft 2 \\ \hline \hline
%6    & Mar 12 &5.4            & Phase Transitions of Mixtures                                                    &                                           & D17, G11     & Draft 2 \\ \hline
%7    & Mar 15 &5.5, 6.3-6.4   & Dilute Solutions; Equipartition; Maxwell Speed Distribution                                          &                                           & D18, G12, W6 & Present \\
%7    & Mar 17 &8.2, Ising.pdf & More about MC; MC Pi estimation, Monte Carlo Simulation Coding; March Madness                                          &                                           & D19          & Present \\
%7    & Mar 19 & PDFs           & The new fluctuation theorems                         &                                           & D20, G13     & Present \\ \hline
%8    & Mar 22 & \multicolumn{3}{c}{FINAL EXAM, Tuesday, March 23 8:00-10:00}                                                                                                    & &n/a          
%\end{tabular}
%\end{table}
%}
\newpage























 
\setlength{\arrayrulewidth}{.4mm}
\setlength{\tabcolsep}{8pt}
{\rowcolors{2}{green!80!yellow!50}{green!70!yellow!40}
  \begin{table}[]
    \footnotesize
\caption{\footnotesize{Course Assignments. Due dates in parentheses}}
\label{tab:course-assignments}
%\begin{tabular}{p{.03\linewidth}|p{.065\linewidth}|p{0.065\linewidth}|p{0.065\linewidth}|p{0.4\linewidth}|p{0.064\linewidth}}
\begin{tabular}{p{0.03\linewidth}|p{0.08\linewidth}|p{0.065\linewidth}|p{0.25\linewidth}|p{0.25\linewidth}|p{0.2\linewidth}}
\textbf{Wk} & \textbf{Assigned} & \textbf{Read}           & \textbf{Daily (3x 1hr)}            & \textbf{Weekly (1x 3 hr)}                                       & \textbf{Group (2x 1 hr)}    \\
1  & Feb 1    & 1.1-1.4        & (\textbf{Due}: Feb 3) 1.7a, 1.8    & (\textbf{Due}: Feb 8) 1.16, 1.20, 1.22 (a,b,c,e - give radius), 1.36 & (\textbf{Due}: Feb 5) 1.17       \\
1  & Feb 3    & 1.5-1.6        & (\textbf{Due}: Feb 5) 1.31, 1.43        &                                                        & (\textbf{Due}: Feb 8) 1.34       \\
1  & Feb 5    & 2.1-2.3        & (\textbf{Due}: Feb 8) 2.5, 2.6          &                                                        &                    \\
2  & Feb 8    & 2.4-2.5        & (\textbf{Due}: Feb 10) 2.16, 2.18, 2.21 & (\textbf{Due}: Feb 15) 2.4, 2.11, 2.17, 2.37                         & (\textbf{Due}: Feb 12) 2.8, 2.19 \\
2  & Feb 10   & 2.6, 3.1       & (\textbf{Due}: Feb 12) 2.29, 2.31, 2.35 &                                                        & (\textbf{Due}: Feb 15) 2.33      \\
2  & Feb 12   & 3.2, 3.3       & (\textbf{Due}: Feb 15) 2.38, 3.3, 3.13  &                                                        &                    \\
3  & Feb 15   & 3.4-3.6        &                           & (\textbf{Due}: Feb 22) 3.6,                                          & (\textbf{Due}: Feb 17) 3.14      \\
3  & Feb 17   & 6.1-6.2        &                           &                                                        & (\textbf{Due}: Feb 19)           \\
3  & Feb 19   & 8.2, Ising.pdf &                           &                                                        &                    \\
4  & Feb 22   & 4.1-4.4        &                           &                                                        &                    \\
4  & Feb 24   & 1.7            &                           &                                                        &                    \\
4  & Feb 26   & 5.1-5.2        &                           &                                                        &                    \\
5  & Mar 1    & 5.3            &                           &                                                        &                    \\
5  & Mar 3    & 5.4            &                           &                                                        &                    \\
5  & Mar 5    & 5.5            &                           &                                                        &                    \\
6  & Mar 8    & 6.3-6.4        &                           &                                                        &                    \\
6  & Mar 10   & 6.5-6.7        &                           &                                                        &                    \\
6  & Mar 12   & PDFs           &                           &                                                        &                    \\
7  & Mar 15   & 7.1-7.2        &                           &                                                        &                    \\
7  & Mar 17   & 7.3            &                           &                                                        &                    \\
7  & Mar 19   & 7.5-7.6        &                           &                                                        &                   \end{tabular}
\end{table}
}
%%%%\vskip.25in\noindent\textbf{Daily Schedule}
%%%\newcounter{hw}
%%%\setcounter{hw}{1}
%%%\newcounter{lab}
%%%\setcounter{lab}{1}
%%%\begin{calendar}{2/1/2021}{17} 
%%%  % Semester starts on 1/16/2013 and last for 14 class weeks. You must
%%%  % always start on a Monday, even if the first day is not a
%%%  % Monday. Use holidays to make up the difference.
%%%\setlength{\calboxdepth}{.3in}
%%%\setlength{\calwidth}{7in}
%%%\TRClass
%%%
%%%% Things that are missing
%%%
%%%% Holidays
%%%\caltexton{10}{\framebox{\textbf{Guest Lecture}}}
%%%\caltexton{21}{\begin{framed}{\bf Project Topics Due}\end{framed}}
%%%\caltexton{22}{\begin{framed}{\bf Project Paragraph Due}\end{framed}}
%%%
%%%\Holiday{1/9/2017}{Winter break}
%%%\Holiday{1/10/2017}{Winter break}
%%%\Holiday{2/23/2017}{\textit{Early Semester Break\\yay!}}
%%%\Holiday{2/24/2017}{\textit{Early Semester Break\\yay!}}
%%%\Holiday{2/25/2017}{\textit{Early Semester Break\\yay!}}
%%%\Holiday{2/26/2017}{\textit{Early Semester Break\\yay!}}
%%%
%%%\Holiday{3/18/2017}{\textit {Spring Break}}
%%%\Holiday{3/19/2017}{\textit {Spring Break}}
%%%\Holiday{3/20/2017}{\textit {Spring Break}}
%%%\Holiday{3/21/2017}{\textit {Spring Break}}
%%%\Holiday{3/22/2017}{\textit {Spring Break}}
%%%\Holiday{3/23/2017}{\textit {Spring Break}}
%%%\Holiday{3/24/2017}{\textit {Spring Break}}
%%%\Holiday{3/25/2017}{\textit {Spring Break}}
%%%\Holiday{3/26/2017}{\textit {Spring Break}}
%%%
%%%\Holiday{4/18/2017}{\textit{Campuswide EPIC Expo}}
%%%
%%%\Holiday{4/29/2017}{Reading Day} 
%%%\Holiday{4/30/2017}{Reading Day} 
%%%\Holiday{5/1/2017}{Finals: comprehensive but with emphasis on
%%%  later topics} 
%%%\Holiday{5/2/2017}{Reading Day} 
%%%\Holiday{5/3/2017}{Finals} 
%%%\Holiday{5/4/2017}{Finals} 
%%%% ... and so on
%%%
%%%\caltext{3/30/2017}{\framebox{\bf{Last day to drop}}}
%%%\caltext{4/29/2017}{\framebox{\bf{Last day}}}
%%%\caltexton{18}{\begin{framed}
%%%      \bf{You may begin Lab  \#\arabic{lab}\stepcounter{lab}:
%%%        Entropy of }
%%%      \bf{Unknotting at any point after this}
%%%      \bf{lecture.}
%%%\end{framed}}
%%%
%%%\caltexton{10}{\begin{framed}
%%%    \bf{First test through \S6.2, due at beginning of next Tuesday's class}
%%%\end{framed}}
%%%
%%%\caltexton{23}{\begin{framed}\bf{Second test}\end{framed}}
%%%
%%%
%%%\caltexton{1}{
%%%Read through Schroeder p. 28 (\S 1.1-1.4) 
%%%\\\line(1,0){3}\\
%%%  What is Statistical Physics?; Thermal equilibrium; Microscopic model of ideal gas; equipartition theorem; heat and work 
%%%\\\line(1,0){3}\\
%%%Problems in class: 1.4, 1.14, 1.18 
%%%\\\line(1,0){3}\\
%%%HW \#\arabic{hw}\stepcounter{hw}: 1.7(a), 1.8, % section 1.1
%%%    1.16, 1.17, 1.20 % section 1.2
%%%}
%%%
%%%\caltextnext{
%%%Read through Schroeder p. 48 (\S 1.5-1.7)
%%%\\\line(1,0){3}\\
%%%  Compressive work; Heat capacities;
%%%      Rates of processes
%%%\\\line(1,0){3}\\ 
%%%Problems in class: 1.37, 1.45
%%%\\\line(1,0){3}\\ 
%%%HW \#\arabic{hw}\stepcounter{hw}:
%%%      1.22 (a,b,c,e - give radius), \\ % section 1.2    
%%%      1.31, 1.34, 1.36, % section 1.5 
%%%      1.43 %section 1.6
%%%}
%%%
%%%\caltextnext{
%%%Read through p. 59 (\S 2.1-2.3)
%%%\\\line(1,0){3}\\
%%%Two-State Systems; Einstein model of a solid; Interacting systems
%%%\\\line(1,0){3}\\
%%%Problems in class: Class choice
%%%\\\line(1,0){3}\\
%%%HW \#\arabic{hw}\stepcounter{hw}: 2.4, 2.5, 2.6, 2.8
%%%}
%%%
%%%\caltextnext{
%%%Read through p. 73(\S 2.4-2.5) 
%%%\\\line(1,0){3}\\
%%%Large Systems; Ideal Gas
%%%\\\line(1,0){3}\\
%%%Problems in class: One of the below. Class votes.
%%%\\\line(1,0){3}\\
%%%HW \#\arabic{hw}\stepcounter{hw}:  2.11, 2.16, 2.17, 2.18, 2.19, 2.21
%%%}
%%%
%%%\caltextnext{
%%%Read through p. 92 (\S 2.6, 3.1)
%%%\\\line(1,0){3}\\
%%%ENTROPY!; Temperature
%%%\\\line(1,0){3}\\
%%%Problems in class: class choice!
%%%\\\line(1,0){3}\\
%%%HW \#\arabic{hw}\stepcounter{hw}: 2.29, 2.31, 2.33, 2.35, 2.37
%%%}
%%%
%%%\caltextnext{
%%%Read through p. 107 (\S 3.2, 3.3)
%%%\\\line(1,0){3}\\
%%%Entropy and Heat; Paramagnetism
%%%\\\line(1,0){3}\\
%%%Problems in class: class choice!
%%%\\\line(1,0){3}\\
%%%HW \#\arabic{hw}\stepcounter{hw}: 2.38, 3.3, 3.6, 3.13, 3.14\\
%%%Additional problem from class.
%%%}
%%%
%%%\caltextnext{
%%%Read through p. 121 (\S 3.4, 3.5, 3.6)
%%%\\\line(1,0){3}\\
%%%Mechanical Equilibrium and Pressure; Diffusive
%%%      Equilibrium and Chemical Potential
%%%\\\line(1,0){3}\\
%%%Problems in class: class choice!
%%%\\\line(1,0){3}\\
%%%HW \#\arabic{hw}\stepcounter{hw}: 3.24, 3.30, 3.32, 3.35, 3.36a
%%%}
%%%
%%%\caltextnext{
%%%Read through p. 220-237 (\S 6.1-6.2)
%%%\\\line(1,0){3}\\
%%%The Boltzmann Factor, Average values
%%%\\\line(1,0){3}\\
%%%Problems in class: class choice!
%%%\\\line(1,0){3}\\
%%%HW \#\arabic{hw}\stepcounter{hw}: 6.3 (it's easier to define some
%%%dimensionless variable $t=kT/\epsilon$ and plot $Z(t)$), 6.4, 6.11, 6.12, 6.13,
%%%6.22ab\\Extra Credit: the rest of 6.22 (we'll do the rest of the
%%%problem in class, so you can earn extra credit only by bringing this
%%%to class finished)
%%%}
%%%
%%%% Here's the IPython notebook from class:
%%%% http://nbviewer.ipython.org/url/mglerner.com/375/Basic%2520Python%2520Intro%2C%2520Starting%2520on%2520Monte%2520Carlo.ipynb 
%%%
%%%% The homework is to work your way through everything up to and
%%%% including the "More on Lists" section of
%%%% http://en.wikibooks.org/wiki/Non-Programmer%27s_Tutorial_for_Python_2.6 
%%%
%%%% See MonteCarlo/Ising/Schroeder_2d_ising.py
%%%
%%%% See HW 11 NB before HW 10.
%%%
%%%\caltextnext{
%%%Read through p. 327-356 (\S 8.2)
%%%\\\line(1,0){3}\\
%%%Ising models
%%%\\\line(1,0){3}\\
%%%Problems in class: class choice!
%%%\\\line(1,0){3}\\
%%%HW \#\arabic{hw}\stepcounter{hw}: 8.15, 8.17, 8.25, 8.26 %\\ March
%%%%Madness 1.1, 1.2, 1.3 (see website)
%%%}
%%%
%%%% MGL more information about Ising models!
%%%% Skip 8.1 entirely
%%%\caltextnext{
%%%Read additional assigned material (Ising.pdf) and \S8.2
%%%\\\line(1,0){3}\\
%%%Continue \S8.2, more about MC; MC Pi estimation, Monte Carlo Simulation Coding;
%%%      March Madness code.
%%%\\\line(1,0){3}\\
%%%Problems in class: %8.27, simple MC simulations
%%%\\\line(1,0){3}\\
%%%HW \#\arabic{hw}\stepcounter{hw}: 8.16, 8.18, 8.23
%%%%Finish MC code \\
%%%%March Madness 2.1, 2.2 (see website)
%%%}
%%%
%%%% http://www.mglerner.com/blog/?p=16
%%%%
%%%%You log in by going to: http://statmech.mayhem.cbssports.com/e?ttag=13_cbspaste
%%%%
%%%%I've made accounts for each of you.
%%%%
%%%%Aislinn: your "official" email account is earlham.statmech.1@gmail.com ... the password for the email account, for your CBS Sports Account, and for the Stat Mech bracket pool are all "Boltzmann" (no quotes).
%%%%
%%%%Jacor: your "official" email account is earlham.statmech.2@gmail.com .. same password as above.
%%%%
%%%%David: earlham.statmech.3
%%%%
%%%%Justin: earlham.statmech.4
%%%%
%%%%
%%%% - Does it behave as expected if you vary Temperature, etc.?
%%%% - Make up some new energy functions. Make at least one based on rankings that you find on the web somewhere. If you write your energy function properly, you can plug in things like "team strength" instead of ranking (e.g. you could go to kenpom.com and plug in the numbers from the "Pyth" column).
%%%%- How often does the 2nd place team win for different energy functions and different temperatures (see the new "printwinpercentages" function)
%%%%
%%%%And a relatively fast homework problem: we all understand the Metropolis algorithm by now. The code for the playgame function was actually *more* complicated when I used it directly. We discussed making moves in bracket space. The Metropolis algorithm is well-suited to this: we take a bracket, consider the potential impacts of flipping a certain game, and let the Metropolis algorithm tell us whether or not to flip the game.
%%%%
%%%%However, at the current time, we're just trying to generate a single bracket. In this case, we just want to play a bunch of individual games. If you let Pa be the probability that team "a" wins, and Pb be the probability that team "b" wins, (1) what's the ratio of Pa to Pb? (2) what's Pa? (3) does the attached code match your answer from part 2?
%%%%
%%%%And, just for fun, here's one more version of the code. If you're playing around with it, you may find it easier to pass temperature as an argument to the various functions, rather than setting it directly. So, the attached version has two changes
%%%%
%%%%1) Temperature is now an argument, rather than a global variable.
%%%%2) The code at the end is all under a magic-seeming line that reads "if __name__ == '__main__'". That's a bit of Python magic that makes the code run only if it's executed from the command line, by clicking on the file, or by typing "run MonteCarloBrackets5.py" in iPython.
%%%
%%%% Also look for ExtractKenpomData.py
%%%
%%%\caltextnext{
%%%Read through p. 122-148 (\S 4.1-4.4)
%%%{\bf More than most days, you must have done the reading ahead of class}
%%%\\\line(1,0){3}\\
%%%      Heat Engines and Refrigerators (\S4.1-4.2) \\
%%%      For discussion, but not as important: \S 4.3-4.4
%%%\\\line(1,0){3}\\
%%%Problems in class: class choice!
%%%\\\line(1,0){3}\\
%%%HW \#\arabic{hw}\stepcounter{hw}: 4.7, 4.8, 4.12, 4.14
%%%}
%%%\caltextnext{}
%%%
%%%\caltextnext{
%%%Read through p. (\S 1.7)
%%%\\\line(1,0){3}\\
%%%Diffusion, rates
%%%\\\line(1,0){3}\\
%%%Problems in class: class choice!
%%%\\\line(1,0){3}\\
%%%HW \#\arabic{hw}\stepcounter{hw}: 1.56, 1.68 (hint: you can make life
%%%easier by reading page 47 and assuming that the perfume has spread to
%%%half of the room), report on one
%%%interesting topic from Sethna. March Madness Monte Carlo problems 1-3 (see the github site)
%%%\\Extra credit: finish 1.57, 
%%%}
%%%
%%%
%%%\caltextnext{
%%%Read through p. 149-165 (\S 5.1-5.2)
%%%\\\line(1,0){3}\\
%%%Free energy available as work; Free Energy as a force towards equilibrium
%%%\\\line(1,0){3}\\
%%%Problems in class: 5.7, class choice!
%%%\\\line(1,0){3}\\
%%%HW \#\arabic{hw}\stepcounter{hw}: 5.4, 5.8, 5.9, 1.40, March Madness Monte Carlo problems 4-5.
%%%}
%%%
%%%\caltextnext{
%%%Read through p. 166-185 (\S 5.3)
%%%\\\line(1,0){3}\\
%%%Phase Transformations of Pure Substances
%%%\\\line(1,0){3}\\
%%%Problems in class: class choice!
%%%\\\line(1,0){3}\\
%%%HW \#\arabic{hw}\stepcounter{hw}: 5.26, 5.32, 5.48, 5.52 \\
%%%Extra credit: 5.51
%%%}
%%%
%%%
%%%\caltextnext{
%%%{\bf Read lab handout} % MGL: Brownian motion boltzman constant.pdf
%%%                       % and BoltzmannAndDiffusion.pdf
%%%\\\line(1,0){3}\\
%%%{\bf Lab \#\arabic{lab}\stepcounter{lab}: Diffusion\& modern microscopy}
%%%\\\line(1,0){3}\\
%%%Problems in class: start analysis!
%%%\\\line(1,0){3}\\
%%%HW \#\arabic{hw}\stepcounter{hw}: finish analysis
%%%}
%%%
%%%
%%%
%%%
%%%
%%%\caltextnext{
%%%Read through p. 186-199 (\S 5.4)
%%%\\\line(1,0){3}\\
%%%Phase Transitions of Mixtures
%%%\\\line(1,0){3}\\
%%%Problems in class: class choice!
%%%\\\line(1,0){3}\\
%%%HW \#\arabic{hw}\stepcounter{hw}: 5.35
%%%}
%%%
%%%
%%%% MGL extra material
%%%%\caltextnext{
%%%%Read provided additional material
%%%%\\\line(1,0){3}\\
%%%%Order parameters and critical exponents
%%%%\\\line(1,0){3}\\
%%%%Problems in class: class choice!
%%%%\\\line(1,0){3}\\
%%%%HW \#\arabic{hw}\stepcounter{hw}:
%%%%}
%%%
%%%
%%%
%%%\caltextnext{
%%%Read through p. 200-207, 238-246 (\S 5.5, \S 6.3-6.4)
%%%\\\line(1,0){3}\\
%%%Dilute Solutions; Equipartition; Maxwell Speed Distribution
%%%\\\line(1,0){3}\\
%%%Problems in class: class choice!
%%%\\\line(1,0){3}\\
%%%HW \#\arabic{hw}\stepcounter{hw}: 5.75, 5.76, 5.82, 6.31, 6.38
%%%\\ Extra Credit: 5.81, 6.39
%%%}
%%%
%%%\caltextnext{
%%%Read through p. 247-256 (\S 6.5-6.7)
%%%\\\line(1,0){3}\\
%%%Partition Functions, Free Energy and Composite Systems
%%%      Also catch up
%%%\\\line(1,0){3}\\
%%%Problems in class: class choice!
%%%\\\line(1,0){3}\\
%%%HW \#\arabic{hw}\stepcounter{hw}: work on your papers\\
%%%Extra credit: 6.43, 6.48, 6.53(!)
%%%}
%%%
%%%\caltextnext{
%%%{\bf Read lab handout} 
%%%\\\line(1,0){3}\\
%%%{\bf Lab \#\arabic{lab}\stepcounter{lab}: Simulation of free energy 1}
%%%\\\line(1,0){3}\\
%%%Problems in class: lab!
%%%\\\line(1,0){3}\\
%%%HW \#\arabic{hw}\stepcounter{hw}: Finish lab!
%%%}
%%%
%%%
%%%\caltextnext{
%%%Read provided additional material
%%%\\\line(1,0){3}\\
%%%Student choice: The new fluctuation theorems {\it or} project workday.
%%%\\\line(1,0){3}\\
%%%Problems in class: class choice!
%%%\\\line(1,0){3}\\
%%%HW \#\arabic{hw}\stepcounter{hw}:\\
%%%Extra credit: Jarzynski problem from Tuckerman.
%%%}
%%%
%%%\caltextnext{
%%%{\bf Read lab handout} 
%%%We'll be working through the ``Stretching Deca-alanine'' tutorial from
%%%the Computational Biophysics folks at UIUC. We'll work through the
%%%in-class portions in class, but you'll need to read the three emailed
%%%PDFs ahead of time. %MGL look fro from:me HW 13 to:aislinn to:vadas
%%%% MGL must set up laptops ahead of time.
%%%\\\line(1,0){3}\\
%%%{\bf Lab \#\arabic{lab}\stepcounter{lab}: Simulation of free energy 2}
%%%\\\line(1,0){3}\\
%%%Problems in class: lab!
%%%\\\line(1,0){3}\\
%%%HW \#\arabic{hw}\stepcounter{hw}: Finish lab!
%%%}
%%%
%%%\caltextnext{
%%%Read through p. 257-270 (\S 7.1-7.2)\\
%%%Class Handout: VariousQMDistributions.PDF
%%%\\\line(1,0){3}\\
%%%The Gibbs Factor; Bosons and Fermions
%%%\\\line(1,0){3}\\
%%%Problems in class: class choice!
%%%\\\line(1,0){3}\\
%%%HW \#\arabic{hw}\stepcounter{hw}: 7.8, 7.10, 7.11ace, 7.13\\
%%%Extra Credit: 7.9, 7.13 the rest, 7.18
%%%}
%%%
%%%\caltextnext{
%%%Read through p. 271-287 (\S 7.3)
%%%\\\line(1,0){3}\\
%%%Degenerate Fermi Gases, Density of States
%%%\\\line(1,0){3}\\
%%%Problems in class: class choice!
%%%\\\line(1,0){3}\\
%%%HW \#\arabic{hw}\stepcounter{hw}: 7.23fg, 7.41 (i.e. ``how lasers work'')
%%%\\Extra Credit: 7.22, 7.23abcde, 7.42 (if you do {\it not} do these
%%%for extra credit, ask Michael for the solutions, as they're required
%%%for the other problems.)
%%%}
%%%
%%%\caltextnext{
%%%Read through p. 271-287 (\S 7.3)
%%%\\\line(1,0){3}\\
%%%Density of States\\catch up
%%%\\\line(1,0){3}\\
%%%Problems in class: class choice!
%%%\\\line(1,0){3}\\
%%%HW \#\arabic{hw}\stepcounter{hw}: work on your papers
%%%\\Extra credit: 7.33, 7.34, 7.35
%%%}
%%%
%%%
%%%% MGL: assume blackbody radiation covered in Modern. It's awesome, though!
%%%%\caltextnext{
%%%%Read through p. 288-306 (\S 7.4)
%%%%\\\line(1,0){3}\\
%%%%Blackbody Radiation
%%%%\\\line(1,0){3}\\
%%%%Problems in class: class choice!
%%%%\\\line(1,0){3}\\
%%%%HW \#\arabic{hw}\stepcounter{hw}:
%%%%}
%%%
%%%
%%%\caltextnext{
%%%Read through p. 307-326 (\S 7.5-7.6)
%%%\\\line(1,0){3}\\
%%%Debye Theory of Solids; Bose-Einstein Condensation
%%%\\\line(1,0){3}\\
%%%Problems in class: class choice!
%%%\\\line(1,0){3}\\
%%%HW \#\arabic{hw}\stepcounter{hw}: work on your papers
%%%\\Extra credit: 7.58, 7.60, 7.64
%%%}
%%%
%%%\caltextnext{
%%%\begin{framed}{\bf PROJECT PRESENTATIONS!}\end{framed}
%%%}
%%%
%%%\caltextnext{
%%%\begin{framed}{\bf PROJECT PRESENTATIONS!}\end{framed}
%%%}
%%%
%%%\caltextnext{
%%%Read through p. 
%%%\\\line(1,0){3}\\
%%%\  
%%%\\\line(1,0){3}\\
%%%Problems in class: class choice!
%%%\\\line(1,0){3}\\
%%%HW \#\arabic{hw}\stepcounter{hw}:
%%%}
%%%
%%%\caltextnext{
%%%Read through p. 
%%%\\\line(1,0){3}\\
%%%\  
%%%\\\line(1,0){3}\\
%%%Problems in class: class choice!
%%%\\\line(1,0){3}\\
%%%HW \#\arabic{hw}\stepcounter{hw}:
%%%}
%%%
%%%
%%%\end{calendar}
\end{document}
